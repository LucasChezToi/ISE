\documentclass{article}

\usepackage[utf8]{inputenc}
\usepackage[frenchb]{babel}
\usepackage[T1]{fontenc}
\usepackage{algorithm2e}

\begin{document}
\fbox{
\begin{algorithm}[H]
%  \SetLine % For v3.9
%  %\SetAlgoLined % For previous releases [?]
  \KwData{\\
    Input
    \begin{itemize}
      \item $i$ flow id
      \item $t$ time
    \end{itemize}
    Local
    \begin{itemize}
      \item $W[]$ limits des suites initialisés à 0
    \end{itemize}
  }  
  \KwResult{
    \begin{itemize}
      \item $W_{i,t}^{last_i}$
    \end{itemize}
  }
  \BlankLine
  \For{$h \in [first_i...last_i]$}{
    \tcp{Initialisation de la suite}
    \tcp{(voir article pour la formule)}
    $w1 \leftarrow W_{i,t}^{h(0)}$\;
    $w2 \leftarrow next(w1)$\;
    \tcp{next() calcule le terme suivant de la suite}
    \tcp{(voir article pour la formule)}
    \tcp{Pour $W_{i,t}^{last_{i,j}^{h}}$ on utilise la valeur de $W[last_{i,j}^{h}]$}
    \While{$w1 != w2$}{
      $w1 \leftarrow w2$\;
      $w2 \leftarrow next(w1)$\;
    }
    $W[h] \leftarrow w1$\;
  }
  \textbf{Return} $W[last_i]$\;
  \caption{Pseudo Code pour le calcul de $W_{i,t}^{last_i}$}
\end{algorithm}
}
\end{document}
